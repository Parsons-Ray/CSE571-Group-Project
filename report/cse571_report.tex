\documentclass[10pt,conference]{IEEEtran}

\usepackage{booktabs} 
\usepackage{graphicx}
\usepackage{subcaption}
\usepackage{caption}
\usepackage{url}
\usepackage{cite}
\usepackage{parcolumns}
\usepackage{listings}
\usepackage{minted}
\usepackage{xcolor}
\usepackage{algorithm2e}

\definecolor{light-gray}{gray}{0.95}

\title{
	Evaluation of Sarsa(\(\lambda\)) Learning Agent 
	}

\author{
	\IEEEauthorblockN{Padraic Cashin \IEEEauthorrefmark{1}, 
	        Ruihao Zhou \IEEEauthorrefmark{2}
		David Lahtinen \IEEEauthorrefmark{3}, 
	}
	\IEEEauthorblockA{
		\IEEEauthorrefmark{1} ASU ID: 1214153888 \\
		\IEEEauthorrefmark{2} ASU ID: 1213439264 \\
		\IEEEauthorrefmark{3} ASU ID: 1207725034 \\
	}
}

% \date{}

\begin{document}
\maketitle

\section{Introduction}

One of the most important reinforcement learning technique in Artificial 
Intelligence is \(Q\)-learning and its goal is to learn itself and get a 
desired policy which will help agent to make best choice under that 
circumstances. \(Q\)-learning does not require a model of the environment and 
can handle problem without requiring adaptations, only with with stochastic 
transitions and rewards. By successfully implement agent in individual project 
3, we have already had a good understanding of this method, however, it’s not 
the best method which means it also has some limitation. For example, it may 
be slower and remains inflexible if alternative routes does not appear. Also, 
policies may not be sufficiently similar. In this group project, we try to 
implement a on-policy algorithm called SARSA\((\lambda)\) and \(Q\)-learning 
into our project three at the same time, and find out what’s the difference 
between these two popular method. In the end, we can get straightforward 
result of this comparison. 

\label{sec:intro}

\section{Background}
\label{sec:background}
	
	The Sarsa algorithm belongs to an On-Policy algorithm for TD-Learning. The 
	speciality of this algorithm compared to Q-Learning is that in the next 
	state, maximum reward doesn’t play an role for updating the Q-values. 
	Instead, it uses the next action and the same policy that determined the 
	original action to get new reward. The name Sarsa actually comes from the 
	four first letter of Q(s, a, r, s', a'). s and a represent original state 
	and action, r is the reward observed in the following state, s', a' are 
	the new state-action pair. As you can see in the following pictures, two 
	action selection steps needed for determining the next state-action pair 
	along with the first. The parameters a and r won’t change as they do in 
	Q-Learning. The nature of the policy's dependence on \(\mathcal{Q}\) determine 
	the convergence of the properties of Sarsa algorithm. For example, one 
	could use \(\epsilon\)-soft or \(\epsilon\)-greedy policies. \cite{sutton18} 
	
	\begin{algorithm}
		\DontPrintSemicolon
		Initialize \(Q(s,a) = 0\) for all \(s \in \mathcal{S}, a \in \mathcal{A}(s)\)\;
		\For{each episode} {
			\(E(s, a) = 0\) for all \(s \in \mathcal{S}, a \in \mathcal{A}(s)\)\;
			Initialize \(S,A\)\;
			\For{step in an episode}{
				Take action \(A\) and observe \(S',R\)\;
				Choose \(A'\) based on \(S'\) and current policy\;
				\(\delta \leftarrow R + \gamma Q(S',A') - Q(S,A)\)\;
				\(E(S,A) \leftarrow (1-\alpha) E(S,A) + 1\)\;
				\For{all \(s \in \mathcal{S}, a \in \mathcal{A}(s)\)} {
					\(Q(s,a) \leftarrow Q(s,a) + \alpha \delta E(s,a)\) \;
					\(E(s,a) \leftarrow \gamma \lambda E(s,a)\) \;
				}
				\(S \leftarrow S'\)\;
				\(A \leftarrow A'\)\;
			}
			Until \(S\) is terminal\;
		}
		\caption{Sarsa\((\lambda)\) Algorithm with Dutch Tracing.  Algorithm
		was provided by Sutton et al. \cite{sutton18}}
		\label{sarsa}
	\end{algorithm}

\section{Sarsa(\(\lambda)\) Implementation}
\label{sec:implementation}

	We chose to start with the codebase provided by UC Berkeley’s Pacman-themed AI 
	tutorial. As the SARSA-lambda algorithm requires an eligibility trace we 
	implemented this as a python dict with default value of 0, same as our Q table. 
	Unlike the Q-table, the eligibility trace must be cleared between episodes, 
	so we implemented a check for final step in the function which updated our 
	Q table to clear the eligibility trace.
    To test and demo the sarsa-lambda agent, we used a test that was built-in, 
	and used by UC berkeley which consisted of a small, gridworld maze for our 
	agent to run around in. What we found was that while the Default q agent 
	first had to random walk to the goal, then to the square in front of the 
	goal, and so on for each episode, the SARSA-lambda agent converged very 
	quickly. After a single random walk to the goal, the Q table updated almost 
	immediately to the optimal policy for the default starting position. While 
	we probably could have improved the results by implementing an 
	\(\epsilon\)-greedy function, our results for the basic set learning rate 
	worked very well.

	\begin{figure}[t]
		\begin{minted}
		[
			frame=lines,
			framesep=2mm,
			baselinestretch=1.2,
			fontsize=\tiny,
			bgcolor=light-gray,
		]{python}
def update(self, state, action, nextState, reward):
    nextAction = self.getAction(nextState)

    delta = reward + (self.discount * 
        self.values[(nextState, nextAction)]) - 
        self.values[(state, action)]
    
    self.eligibility[(state, action)] = (1 - self.alpha) * 
        self.eligibility[(state, action)] + 1

    for k, v in self.values.iteritems():
        trace = self.eligibility[k]
        self.values[k] = v + (self.alpha * delta * trace)
        self.eligibility[k] = trace * self.discount * self.y

    if not self.getLegalActions(nextState):
        self.eligibility = util.Counter()
		\end{minted}
		\caption{Implementation of \texttt{update} for Sarsa\((\lambda)\) agent.  Since \texttt{update} is called
		by the agent each time an aciton is selected, the function implements the inner loop
		of the algorithm detailed in Algorithm \ref{sarsa}.}
		\label{fig:sarsa-agent}
	\end{figure}

\section{Results}
\label{sec:results}
	
	Once the Sarsa\((\lambda)\) agent had been integrated into
	the UC Berkley Pacman framework \cite{ucbai}, \textit{gridword.py} and 
	\textit{pacman.py} were used to test out the efficacy of the Sarsa\((\lambda)\)
	agent compared to the \(Q\)-learning agent already present.  
	
	For the this paper, we evaluated policy/value convergence in \textit{gridworld.py}, using
	the BookGrid world.  Convergence was found by exponentially increasing the
	number of training episodes until the policy became stationary.  The default
	values for each agent were: \(\epsilon = 0.05, \gamma = 0.8, \alpha = 0.2\,
	\lambda = 0.8\) and \(noise = 0.2\).
	
	To evaluate each agent in a more complex environment, we used 
	\textit{pacman.py} with smallGrid world. Each agent was trained for 
	\([100,2000]\) episodes and then evaluated for another 20 episodes.  The 
	default settings for each agent was: \(\epsilon = 0.05, \gamma = 0.8, 
	\alpha = 0.2\) and \(\lambda = 0.8\).
	
	\begin{table}
		\begin{tabular}{|c|c|c|c|}
			\hline
			Agent & Time(s)/100 eps & 50\% Win Rate & Value \\
			& & & convergence \\
			\hline \hline
			Sarsa\((\lambda)\) & 1678.92s & 1000 eps & 256 eps \\
			\hline
			\(Q\)-Learning & 3.54s & 800 eps & 4096 eps \\
			\hline
		\end{tabular}
		\caption{Summary of measured differences between Sarsa\((\lambda)\)
		and \(Q\)-learning agents.  Time per 100 episodes (seconds) was 
		taken from 20 trials using the \textit{pacman.py} smallGrid world.  
		Similarly, the agents were trained on an increasing number of episodes until 
		they achieved a 50\% win rate in the \textit{pacman.py} smallGrid 
		world.  Value convergence was found by exponentially increasing 
		the number of training episodes until the \(Q\)-values and 
		action policy became stationary.  Value convergence was run
		using \textit{gridworld.py} on the BookGrid world.}
		\label{summary}
	\end{table}
			
	Table \ref{summary} shows a summary of the experimental results.  In 
	general we found that the Sarsa\((\lambda)\) agent out performed the 
	\(Q\)-learning agent in the grid world.  Sarsa\((\lambda)\) was able to 
	converge on an optimal policy 3840 training episodes before the \(Q\)-
	learning agent.  On the other hand, the Sarsa\((\lambda)\) agent took approximately
	500\(\times\) longer to train in the \textit{pacman.py} environment, most 
	likely due the number of \((state, action)\) pairs that exist in the 
	environment.  If we are willing to wait for the training rounds, the 
	Sarsa\((\lambda)\) agent was able to reach a consistent 100\% win rate before
	the \(Q\)-learning agent. 

	\begin{figure*}[h]
		\centering
		\begin{subfigure}[b]{0.40\textwidth}
			\includegraphics[width=\textwidth]{./images/qlearning_gridworld_10}
			\caption{\(Q\)-learning agent \(Q\) values}
		\end{subfigure}%
		\begin{subfigure}[b]{0.40\textwidth}
			\includegraphics[width=\textwidth]{./images/sarsa_gridworld_10}
			\caption{Sarsa\((\lambda)\) agent \(Q\) values}
		\end{subfigure}
		\caption{\(Q\) values of the \(Q\)-learning and Sarsa\((\lambda)\) agents 
		after 10 episodes of training.  The Sarsa\((\lambda)\) agent is able
		to evaluate more of the state space than the \(Q\)-learning agent.
		Each agent was run for 10 episodes on the BookGrid world with a noise
		rate of 0.2, and \(\epsilon = 0.5\).}
		\label{qvalues}
	\end{figure*}

	In the \textit{gridworld.py} experiments we found that Sarsa\((\lambda)\) 
	performed better than \(Q\)-learning due its eligibility trace allowing for
	each training episode to update more of the \(Q\)-values than \(Q\)-learning's
	single step functions.  Figure \ref{qvalues} clearly illustrates the difference. 
	After only 10 training episodes, a majority of the \(Q\)-values have been
	updated at least once by the Sarsa\((\lambda)\) agent; while the \(Q\)-learning
	agent has only updated a few of its \(Q\)-values.

	The ability to update the \(Q\)-values associated with each \((state,action)\) 
	in the eligibility trace allows for good decisions be effect the value of the 
	prior decisions, and thus speeds up policy convergence.  It also, means that
	Sarsa\((\lambda)\) is susceptible to noise, as a series of poor or random choices 
	that lead to a positive outcome are \textit{all} updated as though those 
	decisions were good or meaningful to the final outcome.  This leads the agent
	to get temporarily stuck in policies which are not optimal, meaning take longer
	to achieve the goal.

	\begin{figure*}[h]
		\centering
		\begin{subfigure}{0.5\textwidth}
			\includegraphics[width=\textwidth]{./images/qlearning_winrate}
			\caption{\(Q\)-Learning Agent}
		\end{subfigure}%
		\begin{subfigure}{0.5\textwidth}
			\includegraphics[width=\textwidth]{./images/sarsa_winrate}
			\caption{Sarsa\((\lambda)\) Agent}
		\end{subfigure}
		\caption{Win rate vs training episodes for \(Q\)-learning and
		Sarsa\((\lambda)\) agents.  Each agent was trained using 
		pacman smallGrid environment.  Win rate was taken from 
		20 post training episodes. Each agent was trained with 
		\(\epsilon = 0.05, \gamma = 0.8, \alpha = 0.2\).  The 
		Sarsa\((\lambda)\) agent has a default \(\lambda = 0.8\).}
		\label{winrate}
	\end{figure*}

	The other trade off of using the eligibility trace each time the agent 
	picks an action, is the time needed to loop through all of the \(Q\)-values
	in the environment.  For small simple environments such as \textit{gridworld.py},
	the trade off between training time and number of training episodes puts 
	Sarsa\((\lambda)\) ahead of \(Q\)-learning.  However, in the more complex 
	\textit{pacman.py} environment the Sarsa\(\lambda)\) agent slows down 
	considerably.  Even though Figure \ref{winrate} shows that Sarsa\((\lambda)\) 
	was able to reach peak win rate with fewer training episodes than \(Q\)-learning,
	the difference in real world training time was several hours.

\section{Conclusion}
From what we have already implemented, we can easily draw two conclusion:

1. Q-learning maybe suffer from problems converging as a result due to it has higher per-sample variance than SARSA, however SARSA learns a near-optima policy while exploring. So if we decide to use SARSA, we need propose a good strategy to decay ϵ in  ϵ-greedy action choice, which can become fiddly hyper parameter to adjust.

2. SARSA is more conservative than Q-learning, because SARSA will approach convergence allowing for all kinds of penalties while agent explore its move while the Q-learning ignore them. In another words, if there is risk of a large negative reward close to the optimal path, Q-learning will perform more aggressive and will trigger that reward risk. On the contrary, SARAS will try to avoid it and slowly learn to use it.

All in all, Q-learning learn optimal policy directly and SARAS learn online. If you want to train agent in low-cost and fast0iterating environment, choose Q-learning. And if you care about reward risks while learning, SARAS will be the best choice compared Q-learning.

\label{sec:conclusion}

\bibliographystyle{abbrv}
\bibliography{ref}
\end{document}












